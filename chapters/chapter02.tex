\section{First section chap 2} \label{sect_21}
Mr. Bennet was among the earliest of those who waited on Mr.
      Bingley. He had always intended to visit him, though to the last
      always assuring his wife that he should not go; and till the
      evening after the visit was paid she had no knowledge of it. It
      was then disclosed in the following manner. Observing his second
      daughter employed in trimming a hat, he suddenly addressed her
      with:

      “I hope Mr. Bingley will like it, Lizzy.”

      “We are not in a way to know what Mr. Bingley likes,” said her
      mother resentfully, “since we are not to visit.”

      “But you forget, mamma,” said Elizabeth, “that we shall meet him
      at the assemblies, and that Mrs. Long promised to introduce him.”

      “I do not believe Mrs. Long will do any such thing. She has two
      nieces of her own. She is a selfish, hypocritical woman, and I
      have no opinion of her.”

      “No more have I,” said Mr. Bennet; “and I am glad to find that
      you do not depend on her serving you.”

      Mrs. Bennet deigned not to make any reply, but, unable to contain
      herself, began scolding one of her daughters.

      “Don’t keep coughing so, Kitty, for Heaven’s sake! Have a little
      compassion on my nerves. You tear them to pieces.”

      “Kitty has no discretion in her coughs,” said her father; “she
      times them ill.”

      “I do not cough for my own amusement,” replied Kitty fretfully.
      “When is your next ball to be, Lizzy?”

      “To-morrow fortnight.”

      “Aye, so it is,” cried her mother, “and Mrs. Long does not come
      back till the day before; so it will be impossible for her to
      introduce him, for she will not know him herself.”

      “Then, my dear, you may have the advantage of your friend, and
      introduce Mr. Bingley to her.”

      “Impossible, Mr. Bennet, impossible, when I am not acquainted
      with him myself; how can you be so teasing?”

      “I honour your circumspection. A fortnight’s acquaintance is
      certainly very little. One cannot know what a man really is by
      the end of a fortnight. But if we do not venture somebody else
      will; and after all, Mrs. Long and her nieces must stand their
      chance; and, therefore, as she will think it an act of kindness,
      if you decline the office, I will take it on myself.”

      The girls stared at their father. Mrs. Bennet said only,
      “Nonsense, nonsense!”

      “What can be the meaning of that emphatic exclamation?” cried he.
      “Do you consider the forms of introduction, and the stress that
      is laid on them, as nonsense? I cannot quite agree with you
      there. What say you, Mary? For you are a young lady of deep
      reflection, I know, and read great books and make extracts.”

\begin{figure}[h]
	\centering
	\includegraphics{crest_fullcolour.pdf}
	\caption{Example graph}
	\label{plot_example_graph}
\end{figure}
      Mary wished to say something sensible, but knew not how.

      “While Mary is adjusting her ideas,” he continued, “let us return
      to Mr. Bingley.”

      “I am sick of Mr. Bingley,” cried his wife.

      “I am sorry to hear that; but why did not you tell me that
      before? If I had known as much this morning I certainly would not
      have called on him. It is very unlucky; but as I have actually
      paid the visit, we cannot escape the acquaintance now.”

      The astonishment of the ladies was just what he wished; that of
      Mrs. Bennet perhaps surpassing the rest; though, when the first
      tumult of joy was over, she began to declare that it was what she
      had expected all the while.

      “How good it was in you, my dear Mr. Bennet! But I knew I should
      persuade you at last. I was sure you loved your girls too well to
      neglect such an acquaintance. Well, how pleased I am! and it is
      such a good joke, too, that you should have gone this morning and
      never said a word about it till now.”

      “Now, Kitty, you may cough as much as you choose,” said Mr.
      Bennet; and, as he spoke, he left the room, fatigued with the
      raptures of his wife.

      “What an excellent father you have, girls!” said she, when the
      door was shut. “I do not know how you will ever make him amends
      for his kindness; or me, either, for that matter. At our time of
      life it is not so pleasant, I can tell you, to be making new
      acquaintances every day; but for your sakes, we would do
      anything. Lydia, my love, though you are the youngest, I dare
      say Mr. Bingley will dance with you at the next ball.”
	  \begin{table}[h]
	  	\begin{subtable}[h]{0.45\textwidth}
	  		\centering
	  		\begin{tabular}{l | l | l}
	  			Day & Max Temp & Min Temp \\
	  			\hline \hline
	  			Mon & 20 & 13\\
	  			Tue & 22 & 14\\
	  			Wed & 23 & 12\\
	  			Thurs & 25 & 13\\
	  			Fri & 18 & 7\\
	  			Sat & 15 & 13\\
	  			Sun & 20 & 13
	  		\end{tabular}
	  		\caption{First Week}
	  		\label{tab:week1}
	  	\end{subtable}
	  	\hfill
	  	\begin{subtable}[h]{0.45\textwidth}
	  		\centering
	  		\begin{tabular}{l | l | l}
	  			Day & Max Temp & Min Temp \\
	  			\hline \hline
	  			Mon & 17 & 11\\
	  			Tue & 16 & 10\\
	  			Wed & 14 & 8\\
	  			Thurs & 12 & 5\\
	  			Fri & 15 & 7\\
	  			Sat & 16 & 12\\
	  			Sun & 15 & 9
	  		\end{tabular}
	  		\caption{Second Week}
	  		\label{tab:week2}
	  	\end{subtable}
	  	\caption{Max and min temps recorded in the first two weeks of July}
	  	\label{tab:temps}
	  \end{table}
      “Oh!” said Lydia stoutly, “I am not afraid; for though I am the
      youngest, I’m the tallest.”

      The rest of the evening was spent in conjecturing how soon he
      would return Mr. Bennet’s visit, and determining when they should
      ask him to dinner.